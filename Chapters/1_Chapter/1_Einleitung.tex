\section{Einleitung}
Die Musiktheorie gliedert sich in Harmonie- und Tonsatzlehre, Analyse, 
Formenlehre, Gehörbildung, Höranalyse, Improvisation und Komposition.
Musiktheorie erklärt dir, wie Musik funktioniert. Sie ist die Struktur, die den Songs, die 
du liebst, zugrunde liegt und erklärt, wie sie funktionieren. Doch Musiktheorie kann 
auch den Weg nach vorne weisen. Zumindest die Grundlagen der Theorie zu lernen ist ein 
unumgänglicher Teil deiner musikalischen Entwicklung. Am Anfang kann die Theorie etwas 
einschüchternd wirken. Das Thema ist so groß, dass es schwer ist zu wissen, wo man am 
besten anfängt. Ich haben einen Leitfaden zusammengestellt, der dir dabei hilft, mit der Musiktheorie 
loszulegen, sodass du die Grundlagen leicht verstehen und sie korrekt auf deine eigene 
Musik anwenden kannst.

% \gtab{chord name}{fret:strings:fingering}
\gtab{C}{032010:032010} \gtab{Cmaj7}{032000:032000}
\gtab{\raisebox{5pt}{\quad F}}{(133211):043200}
\gtab{\raisebox{5pt}{\quad F#}}{2fr:(133211):034200}
\gtab{D}{000232:000132}