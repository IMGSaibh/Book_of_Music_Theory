\section{Einleitung}
Die Musiktheorie gliedert sich in Harmonie- und Tonsatzlehre, Analyse, 
Formenlehre, Gehörbildung, Höranalyse, Improvisation und Komposition.
Musiktheorie erklärt, wie Musik funktioniert. Sie ist die Struktur, die einem Song zugrunde liegt und erklärt, wie sie funktionieren. Doch Musiktheorie kann 
auch den Weg nach vorne weisen. Zumindest die Grundlagen der Theorie zu lernen ist ein 
unumgänglicher Teil der musikalischen Entwicklung. Am Anfang kann die Theorie etwas 
einschüchternd wirken. Das Thema ist so groß, dass es schwer ist zu wissen, wo man am 
besten anfängt. Dies ist ein Leitfaden, der dabei hilft, mit der Musiktheorie 
loszulegen, sodass die Grundlagen leicht verstanden werden und sie korrekt auf eigene 
Musik angewendet werden können.