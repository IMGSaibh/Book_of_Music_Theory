\section{Grundlagen}
\subsection{Ton und Tonstufe}
    \textbf{Die Tonstufe} ist ein einzelner Ton einer diatonischen Tonleiter. Als Kernbestand des abendländischen 
    Tonsystems kann man die sieben Tonstufen \textbf{C, D, E, F, G, A und H ansehen, die Stammtöne.} Die Tonstufen 
    einer gegebenen Tonleiter werden mit lateinischen Ordinalzahlen benannt.

\begin{table}[H]
    \caption{Tonstufen C-Dur Tonleiter}
    \begin{tabularx}{\textwidth}{|>{\hsize=.2\hsize}X|>{\hsize=.4\hsize}X|>{\hsize=.4\hsize}X|}
    \hline
    Tonstufe & Ton & lateinische Ordinalzahlen\\ \hline
    \textbf{1} & \textbf{C} & \textbf{Prime} \\ \hline
    2 & D & Sekunde \\ \hline
    3 & E & Terz \\ \hline
    4 & F & Quarte \\ \hline
    5 & G & Quinte\\ \hline
    6 & A & Sexte \\ \hline
    7 & H & Septime \\ \hline
    \textbf{8} & \textbf{C} & \textbf{Oktave}\\ \hline   
    9 & D & None\\ \hline   
    10 & E & Dezime\\ \hline   
    11 & F & Undezime\\ \hline   
    12 & G & Doudezime\\ \hline   
    12 & A & Terzdezime\\ \hline   
    \end{tabularx}
\end{table}

\subsection{Tonleitern}
Eine Tonleiter oder Ton-Skala ist in der Musik eine Reihe von der Tonhöhe nach geordneten Tönen, 
jenseits derer die Tonreihe in der Regel wiederholbar ist. In den meisten Fällen hat eine Tonleiter 
den Umfang einer Oktave. Weit verbreitet sind diatonische Tonleitern in Dur und Moll. Tonleitern sind 
durch Tonabstände definiert. An erster Stelle steht der Grundton (wie zum Beispiel C) nach 8 Tönen
(also einer Oktave) wiederholt sich die Reihe. Der 8. Ton ist also wieder ein C. 
\textbf{Die Verbindung des Tongeschlechts (wie Dur oder Moll) mit dem Grundton ergibt die Tonart}. 
C-Dur ist hierfür ein Beispiel. Weitere sind A-Moll, D-Dur usw.

Eine Dur-Tonleiter wird  gebildet, ausgehend vom Grundton, indem jeweils ein Ganztonschritt weiter 
gesprungen wird. Die \textbf{Stellen von Ton 3 zu 4 und vom Ton 7 zu 8.} sind Ausnahmen. Hier wird nur ein
Halbtonschritt gesprungen. Analog wird diese Verfahrensweise auch für alle weiteren Grundtöne angewendet. 
In der Tabelle ist die D Dur-Tonleiter als weiteres Beispiel angegeben. Die Farben markieren den Halbtonschritt.

\begin{table}[H]
    \caption{Tonstufen der Dur-Tonleitern}
    \begin{tabularx}{\textwidth}{|>{\hsize=.1\hsize}X|>{\hsize=.1\hsize}X|>{\hsize=.1\hsize}X|>{\hsize=.1\hsize}X|>{\hsize=.1\hsize}X|>{\hsize=.1\hsize}X|>{\hsize=.1\hsize}X|>{\hsize=.1\hsize}X|>{\hsize=.1\hsize}X|}
    \hline
    1. Ton & 2. Ton & 3. Ton & 4. Ton & 5. Ton & 6. Ton & 7. Ton & 8. Ton \\ \hline  
    C & D & \cellcolor{gray!25}E & \cellcolor{gray!25}F & G & A & \cellcolor{gray!25}H & \cellcolor{gray!25}C \\ \hline  
    D & E & \cellcolor{gray!25}Fis & \cellcolor{gray!25}G & A & H & \cellcolor{gray!25}Cis & \cellcolor{gray!25}D \\ \hline  
    \end{tabularx}
\end{table}

Eine Moll-Tonleiter wird  gebildet, ausgehend vom Grundton, indem jeweils ein Ganztonschritt weiter 
gesprungen wird. Die \textbf{Stellen von Ton 2 zu 3 und vom Ton 5 zu 6.} sind Ausnahmen. Hier wird nur ein
Halbtonschritt gesprungen. Analog wird diese Verfahrensweise auch für alle weiteren Grundtöne angewendet. 
In der Tabelle ist die A und H Dur-Tonleiter als Beispiel angegeben. Die Farben markieren den Halbtonschritt.


\begin{table}[H]
    \caption{Tonstufen der Moll-Tonleitern}
    \begin{tabularx}{\textwidth}{|>{\hsize=.1\hsize}X|>{\hsize=.1\hsize}X|>{\hsize=.1\hsize}X|>{\hsize=.1\hsize}X|>{\hsize=.1\hsize}X|>{\hsize=.1\hsize}X|>{\hsize=.1\hsize}X|>{\hsize=.1\hsize}X|>{\hsize=.1\hsize}X|}
    \hline
    1. Ton & 2. Ton & 3. Ton & 4. Ton & 5. Ton & 6. Ton & 7. Ton & 8. Ton \\ \hline
    A & \cellcolor{gray!25}H & \cellcolor{gray!25}C & D & \cellcolor{gray!25}E & \cellcolor{gray!25}F & G & A \\ \hline
    H & \cellcolor{gray!25}Cis & \cellcolor{gray!25}D & E & \cellcolor{gray!25}Fis & \cellcolor{gray!25}G & A & H \\ \hline
    \end{tabularx}
\end{table}


\subsection{Akkord}
\subsection{Stufenakkorde}
Für die Stufenakkorde wird ebenfalls vom Akkord der ersten Stufe ausgehend ein ein ganzer Akkord weiter gesprungen. \textbf{Die Ausnahme bilden hier wieder die Stufen 3 und 4} hier wieder die Stufen 3 und 4 bei denen ein Halbtone weiter gesprungen wird. Die Stufen 7 und 8 werden nicht betrachtet, weil auf der Stufe 8 wieder der Akkord der ersten Stufe steht. 
\begin{table}[H]
    \caption{Akkord und Stufenakkorde}
    \begin{tabularx}{\textwidth}{|>{\hsize=.3\hsize}X|>{\hsize=.3\hsize}X|>{\hsize=.3\hsize}X|>{\hsize=.3\hsize}X|>{\hsize=.3\hsize}X|>{\hsize=.3\hsize}X|>{\hsize=.3\hsize}X|}
    \hline
    \RomanNumeralCaps{1} & \RomanNumeralCaps{2} & \RomanNumeralCaps{3} & \RomanNumeralCaps{4} & \RomanNumeralCaps{5} & \RomanNumeralCaps{6} & \RomanNumeralCaps{7} \\ \hline
    C & D & \cellcolor{gray!25}E & \cellcolor{gray!25}F & G & A & H \\ \hline
    D & E & \cellcolor{gray!25}Fis & \cellcolor{gray!25}G & A & H & Cis \\ \hline
    \end{tabularx}
\end{table}

\subsection{Melodie}
Eine Melodie ist eine \textbf{lineare Sequenz von Tönen}, die von der Person, 
die sie sich anhört, als Einheit wahrgenommen wird. Die Melodie steht im Vordergrund 
eines Songs und besteht aus einer \textbf{Kombination aus Tonhöhe und Rhythmus}. Tonsequenzen, 
die eine Melodie ausmachen, sind musikalisch befriedigend und meistens der einprägsamste 
Teil eines Songs.

Von eingängigen Refrains bis hin zu mitreißenden Gitarrenriffs – Melodien definieren die Musik, 
weil sie Teil der Musik sind, die am ehesten im Gedächtnis bleibt. Dabei ist zu beachten, dass 
es einen Unterschied gibt zwischen Harmonie 
und Melodie: \textbf{Eine Melodie wird zur Harmonie, wenn vollkommen unterschiedliche Töne darüber- oder 
daruntergesetzt werden und alles zur gleichen Zeit gespielt wird}. Auf diese Weise werden Akkorde 
sowie Gesangs- und Instrumental-Harmonien gebildet. Eine gute Melodie zieht die Aufmerksam der 
Zuhörer*innen auf sich und fesselt sie. Songwriter*innen und Komponist*innen nutzen Melodien, 
um Geschichten zu erzählen und dem Publikum etwas zu geben, an das sie anknüpfen und sich erinnern 
können. Die gängigsten Arten, Melodien in der Musik einzusetzen, sind Strophen, Refrains und Bridge-Vocals, 
doch auch Instrumental-Melodien sind wichtig.

Selbst die simpelste Melodie profitiert von unerwarteten Rhythmen. Hier sind ein paar Tipps um Melodien 
eindringlicher und abwechselungsreicher zu gestalten:
\begin{itemize}
    \item \textbf{simpelste Melodien profitieren von unerwarteten Rhythmen}
    \item \textbf{Wenn Melodien immer bei Beat 1 beginnen}: Kann sie kurz davor oder danach beginnen. Selbst die 
    kleinste Veränderung im Rhythmus kann eine Melodie auf subtile, jedoch massive Weise verändern.
    \item \textbf{Melodische Kontur}: Ist die allgemeine Form einer Melodie, d.h. die Art, wie sie sich nach 
    oben und unten bewegt.
    \item \textbf{Bewegung in Melodie}: Bewegung in Sprüngen findet statt, wenn eine Melodie in Intervallen, die 
    größer sind als eine Sekunde, fortschreitet. Allerdings sind zu viele große Sprünge hintereinander schwieriger 
    in eine einzelne melodische Einheit zusammenzufassen.
    \item \textbf{Melodien existieren nicht in einem Vakuum}: Es besteht ein wichtiges Gleichgewicht zwischen der 
    Melodie und ihrer unterschwelligen Harmonie. Die Akkordnoten \textbf{(die Tonleiterschritte 1, 3, 5 und 7)} sind die 
    stärksten und \textbf{stabilsten Plätze, auf denen man landen kann}.
    
\end{itemize}


\subsection{Harmonie}
Harmonien sind wichtige Elemente eines Songs. Sie ist allerdings nicht immer leicht umzusetzen. 
Doch es gibt ein paar Elemente eines Songs, die das Songwriting voranbringen – die Harmonik gehört dazu. 
Egal ob fortgeschrittene*r Musiker*in oder vollkommener Neuling – musiktheoretisches Wissen bietet massive Vorteile.

Harmonien entstehen immer dann, wenn zwei oder mehrere Töne gleichzeitig gespielt werden. \textbf{Harmonik} kann 
sich auf das \textbf{Arrangement individueller Töne bzw. Tonstufen zu einem Akkord sowie auf die allgemeine Akkordstruktur 
eines Musikstücks beziehen}. Doch in der Musiktheorie bezieht sich Harmonik normalerweise auf das Bilden 
von Akkorden, die Eigenschaften von Akkorden und Akkordfolgen.

Harmonik wird durch römische Ziffern repräsentiert. Der Name eines Akkords wird durch eine römische Ziffer
ersetzt. Und zwar \textbf{die Ziffer die der Tonstufe seines Grundtons entspricht}. Römische Ziffern geben an, 
zu welcher harmonischen Kategorie ein Akkord gehört. Und Akkord-Kategorien bestimmen die Funktion eines 
Akkords innerhalb eines Musikstücks. In der tonalen Musik gibt es drei funktionale Kategorien:

\begin{itemize}
    \item \textbf{Tonika}: Tonika-Akkorde sind Ruhezonen, bei denen einem die harmonische Aktion eines 
    Songs am stabilsten vorkommt.
    \item \textbf{Dominante}: Dominant-Akkorde stellen eine Art Gegenteil zu den Tonika-Akkorden dar. 
    Bei diesem Akkord will die vierte und siebte Tonleiterstufe auf natürliche Weise nach unten 
    (vierte-dritte) und nach oben (siebte-Tonika) rücken, um den Tonika-Akkord aufzulösen.
    \item \textbf{Prädominante}: Die anderen Akkorde in dem harmonischen Vokabular sind die Prädominanten, 
    mit denen die Lücke zwischen Stufe I und Stufe V überbrückt wird.
\end{itemize}

Hier vielleicht der Quintenzirkel und ein Beispiel welche Akkorde gut zusammen passen

\subsection{Rhytmus}
Rhythmus ist einer der fundamentalen Aspekte der Musiktheorie. Um gute Harmonien und Melodien komponieren zu können, musst du verstehen, wie Rhythmus funktioniert und wie du ihn in deinen Tracks einsetzt. Anhand von Rhythmus wird Musik systematisch in Taktschläge bzw. Beats eingeteilt, die sich innerhalb eines Taktes mit einem allgemein anerkannten Tempo wiederholen. Noten, Melodien und Akkorde kann man leicht als Vibrationen in der Luft definieren, die von unserem Trommelfell wahrgenommen werden.

\textbf{Hier  eventuell Takt erklaeren}

Rhythmus hingegen hat eher etwas mit der einzigartigen Wahrnehmung von Zeit zu tun, über die der Mensch verfügt. Zumindest ist das die Definition, die dir ein Metronom geben würde. Rhythmus ist etwas, dass mit der einzigartigen Wahrnehmung von Zeit zu tun, über die der Mensch verfügt. Um Rhythmus zu verstehen, muss man vier grundlegende Konzepte kennen:

\begin{itemize}
    \item \textbf{Taktschläge (=Beats) und Noten}
    \item \textbf{Takte und Taktangaben}
    \item \textbf{Schwache (=unbetonte) und starke (=betonte) Zählzeiten}
    \item \textbf{Zweier- und Dreiertakte}
\end{itemize}

Wenn du diese vier Grundkonzepte beherrschst, kannst du besser üben und neue, interessante Rhythmen in deinen Tracks verwenden. Um zum Zentrum eines Rhythmus vorzudringen, musst du verstehen, dass eine Musiknote die zeitliche Dauer, während der ein Instrument gespielt wird, repräsentiert. 

\subsubsection{Taktschläge und Noten}
Eine ganze Note ist die längste Tondauer, doch ganze Noten können in halbe, Viertel-, Achtel- und Sechzehntelnoten runtergebrochen werden.

\begin{table}[H]
    \caption{Taktschläge und Noten}
    \begin{tabularx}{\textwidth}{| X | X | X |}
    \hline
    Note & Pause & Anschlag Pro Takt \\ \hline
    \semibreve & \wholeNoteRest\ & 1 \\ \hline
    \minim & \halfNoteRest & 2 \\ \hline
    \crotchet & \crotchetRest  & 4 \\ \hline
    \quaver & \quaverRest  & 8 \\ \hline
    \semiquaver & \semiquaverRest & 16 \\ \hline
    \end{tabularx}
\end{table}

Es gibt vieles, dem man sich widmen muss, wenn man verstehen will, wie man musikalische Rhythmen liest.Doch um zum Zentrum eines Rhythmus vorzudringen, musst du verstehen, dass eine Musiknote die zeitliche Dauer, während der ein Instrument gespielt wird, repräsentiert. Eine ganze Note ist die längste Tondauer, doch ganze Noten können in halbe, Viertel-, Achtel- und Sechzehntelnoten runtergebrochen werden. Eine halbe Note nimmt lediglich die halbe Dauer einer ganzen Note in Anspruch, eine Viertelnote ein Viertel.


\subsubsection{Takte und Taktangaben}
Jeder Musik liegt ein Puls zugrunde, der in bestimmte Zeiteinheiten gefasst werden kann. Diese Zeiteinheit werden als Takte bezeichnet. In der westlichen Musik geben die Taktangaben eines Songs vor, wie sein Puls in jedem Takt gemessen wird, und das Tempo legt fest, wie schnell der Puls ist. Nehmen wir die gängigste Taktart in der Musik – den 4/4-Takt.
Die obere Vier gibt an, dass es vier Taktschläge in einem Takt gibt, und die untere Vier gibt an, dass alle Taktschläge in Viertelnoten gemessen werden.

\begin{figure}[H]
    \centering
    \includegraphics[width=0.8\textwidth]{images/Rythm_body}
\end{figure}

Natürlich gibt es noch viel mehr Taktarten als den 4/4-Takt. Alle Walzer sind im ¾-Takt und dann gibt es auch noch die Geschichte mit zusammengesetzten und ungeraden Taktarten.

\subsubsection{Starke und schwache Zählzeiten}
Ok, jetzt da du weißt, wie Taktarten funktionieren und Taktschläge in einen Takt passen, sollten wir uns anschauen, wie Rhythmus innerhalb eines Taktes funktioniert. Innerhalb eines Taktes gibt es starke Taktschläge, die den Puls vorantreibenund schwache Taktschläge,die dem Puls entgegenwirken. Im 4/4-Takt zum Beispiel fallen die starken Taktschläge auf die erste und dritte Viertelnote im Takt, und die schwachen Taktschläge fallen auf die zweite und vierte Viertelnote.

\begin{figure}[H]
    \centering
    \includegraphics[width=0.8\textwidth]{images/Rythm_body_2}
\end{figure}

Im ¾-Takt fällt der starke Taktschlag auf die erste Viertelnote und die schwachen Taktschläge fallen auf die zweite und dritte Viertelnote.

\begin{figure}[H]
    \centering
    \includegraphics[width=0.8\textwidth]{images/Rythm_body_3}
\end{figure}

Wenn du dich erstmal mit starken und schwachen Taktschlägen beschäftigt hast, hörst du sie überall. Beispiele dafür sind der EINS-zwei, EINS-zwei Puls einer Kick-Drum in einem 4/4-Disco-Song oder das EINS-zwei-drei, EINS-zwei-drei eines Walzers.

\subsubsection{Zweier- und Dreiertakte}
Bisher haben wir uns nur mit dem ¾- und 4/4-Takt, den beiden gängigsten Taktarten, beschäftigt. Falls du zusammengesetzte und ungerade Taktarten in deinem Track verwenden willst, musst du verstehen, dass Taktschläge innerhalb eines Taktes in Zweier- und Dreierpaaren gefühlt werden. Das Ganze macht mehr Sinn, wenn du weißt, wie starke und schwache Taktschläge funktionieren. Eine Art, Dreier- und Zweiertakte zu visualisieren, besteht darin, sich den Unterschied zwischen einem rollenden Dreieck und einem rollenden Quadrat vorzustellen, bei dem mit jeder neuen Umdrehung ein starker Taktschlag fällt.

Wenn du dir die starken und schwachen Taktschläge in einem 4/4-Takt anschaust, kannst du sie in zwei Zweier-gruppen unterteilen – stark – schwach, stark – schwach.

\begin{figure}[H]
    \centering
    \includegraphics[width=0.8\textwidth]{images/Rythm_body_4}
\end{figure}

Ein stark-schwach-Muster bedeutet, dass es sich um einen Zweiertakt handelt. Da er in zwei Zweierpaare unterteilt ist, wird der 4/4-Takt manchmal auch Quadrupel-Takt genannt. Im ¾-Takt gibt es nur eine Dreiergruppe – stark – schwach – schwach.

\begin{figure}[H]
    \centering
    \includegraphics[width=0.8\textwidth]{images/Rythm_body_5}
\end{figure}

Ein stark-schwach-schwach-Muster bedeutet, dass es sich um einen Dreiertakt handelt. Jedes rhythmische Muster und jeder Takt kann in Zweier- oder Dreiermetren unterteilt werden.

\subsubsection{Simple vs. zusammengesetzte Taktarten}
Simple und zusammengesetzte Taktarten hängen direkt mit dem Metrum zusammen. Das Metrum legt fest, wie der Rhythmus anhand von starken und schwachen Taktschlägen gefühlt wird. Simple und zusammengesetzte Taktarten bestimmen, ob eine Einheit aus kürzeren Noten (meistens Achtelnoten) in Zweier- oder Dreiergruppen aufgeteilt wird. Simple Taktarten gruppieren Achtelnoten in Zweiergruppen. Der 4/4-Takt ist ein simpler Zweiertakt. Seine Achtelnoten werden gezählt als EINS-und, zwei-und, DREI-und, vier-und.

\begin{figure}[H]
    \centering
    \includegraphics[width=0.8\textwidth]{images/Rythm_body_6}
\end{figure}

Der ¾-Takt ist ein simpler Dreiertakt. Er wird gezählt als EINS-und, zwei-und, drei-und.

\begin{figure}[H]
    \centering
    \includegraphics[width=0.8\textwidth]{images/Rythm_body_7}
\end{figure}

Zusammengesetzte Taktarten gruppieren Achtelnoten in Dreiergruppen. 6/8 und 9/8 sind beide Beispiele für eine zusammengesetzte Taktart. Im zusammengesetzten 6/8-Zweiertakt werden die Noten in zwei Gruppen aus jeweils drei Achtelnoten unterteilt.

\begin{figure}[H]
    \centering
    \includegraphics[width=0.8\textwidth]{images/Rythm_body_8}
\end{figure}

Die Achtelnoten könnten somit als EINS-und-a, ZWEI-und-a gezählt werden. Drakes Song Plastic Bag ist ein tolles Beispiel für einen Pop-Song, der dem 6/8-Rhythmus folgt. Im zusammengesetzten 9/8-Dreiertakt sind die Noten in drei Gruppen aus drei Achtelnoten unterteilt.

\begin{figure}[H]
    \centering
    \includegraphics[width=0.8\textwidth]{images/Rythm_body_9}
\end{figure}

Die Achtelnoten werden gezählt als EINS-und-a, ZWEI-und-a, DREI-und-a. Der berühmte Jazz-Track Blue Rondo A La Turk von Dave Brubeck ist ein gutes Beispiel für den 9/8-Takt. Der Rhythmus in diesem Track wechselt zwischen dem zusammengesetzten und ungeraden 9/8-Takt. Finde heraus, ob den den Unterschied hören kannst!

\subsubsection{Ungerade Taktarten}

Ungerade Takte können etwas einschüchternd sein, es gibt viel zu wissen auf diesem Gebiet. Doch wenn du erstmal weißt, wie Zweier- und Dreiermetren funktionieren, kommst du auch mit jedem ungeraden Takt klar. Ungerade Taktarten gehen noch viel weiter mit den Regeln als simple und zusammengesetzte Taktarten, indem sie sie kombinieren. Das liegt daran, dass alle ungeraden Taktarten einem Muster folgen, das auf einer Kombination aus Zweier- und Dreiergruppen basiert. Schauen wir uns den ⅝-Takt an. Er kann entweder aus einer Zweiergruppe plus einer Dreiergruppe bestehen, oder einer Dreiergruppe plus einer Zweiergruppe.

\begin{figure}[H]
    \centering
    \includegraphics[width=0.8\textwidth]{images/Rythm_body_10}
\end{figure}

Falls das deiner Meinung nach keinen Sinn ergibt, versuche es einfach damit, das Metrum laut mitzuzählen, jedoch nur in Dreier- und Zweiergruppen. Für einen ⅝-Takt würdest du dementsprechend entweder EINS-und ZWEI-und-a zählen, oder EINS-und-a ZWEI-und. Wenn wir uns wieder Blue Rondo A La Turk als Beispiel ansehen, dann folgt der 9/8-Teil als ungerader Takt dem Muster EINS-und, ZWEI-und, DREI-und, VIER-und-a.

\begin{figure}[H]
    \centering
    \includegraphics[width=0.8\textwidth]{images/Rythm_body_11}
\end{figure}