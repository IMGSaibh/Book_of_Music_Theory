\subsection{Stufenakkorde}
Für die Stufenakkorde wird ebenfalls vom Akkord der ersten Stufe ausgehend ein ein ganzer Akkord weiter gesprungen. \textbf{Die Ausnahme bilden hier wieder die Stufen 3 und 4} bei denen ein Halbton weiter gesprungen wird. Die Stufen 7 und 8 werden nicht betrachtet, weil auf der Stufe 8 wieder der Akkord der ersten Stufe steht. Jede Stufe wird entweder mit als Dur-Akkord oder Moll-Akkord gespielt. Welche der Akkorde als Dur oder Moll gespielt wird kannst du in dem Tabellenkopf der Tabelle \ref{tab:Tabelle_Stufenakkorde} entnehmen. Akorde die als Moll gespielt werden, sind bei der römischen Nummerierung mit einem anghängten -m- gekennzeichnet. Wie beispielsweise die Stufe \RomanNumeralCaps{2}m. Stufen ohne -m- werden als Dur Akkord gespielt. 
 
\begin{table}[H]
    \caption{Akkord und Stufenakkorde}
    \label{tab:Tabelle_Stufenakkorde}
    \begin{tabularx}{\textwidth}{|>{\hsize=.3\hsize}X|>{\hsize=.3\hsize}X|>{\hsize=.3\hsize}X|>{\hsize=.3\hsize}X|>{\hsize=.3\hsize}X|>{\hsize=.3\hsize}X|>{\hsize=.3\hsize}X|}
    \hline
    \RomanNumeralCaps{1} & \RomanNumeralCaps{2}m & \RomanNumeralCaps{3}m & \RomanNumeralCaps{4} & \RomanNumeralCaps{5} & \RomanNumeralCaps{6}m & \RomanNumeralCaps{7}dim \\ \hline
    C & D & \cellcolor{gray!25}E & \cellcolor{gray!25}F & G & A & B \\ \hline
    D & E & \cellcolor{gray!25}Fis & \cellcolor{gray!25}G & A & B & Cis \\ \hline
    E & Fis & \cellcolor{gray!25}Gis & \cellcolor{gray!25}A & B & Cis & Dis \\ \hline
    F & G & \cellcolor{gray!25}A & \cellcolor{gray!25}Bb & C & D & E \\ \hline
    G & A & \cellcolor{gray!25}B & \cellcolor{gray!25}C & D & E & Fis \\ \hline
    A & B & \cellcolor{gray!25}Cis & \cellcolor{gray!25}D & E & Fis & Gis \\ \hline
    B & Cis & \cellcolor{gray!25}Dis & \cellcolor{gray!25}E & Fis & Gis & Ais \\ \hline
    Bb & C & \cellcolor{gray!25}D & \cellcolor{gray!25}Eb & F & G & A \\ \hline
    Eb & F & \cellcolor{gray!25}G & \cellcolor{gray!25}Ab & B & C & D \\ \hline
    Ab & B & \cellcolor{gray!25}C & \cellcolor{gray!25}Db & Eb & F & G \\ \hline
    Db & Eb & \cellcolor{gray!25}F & \cellcolor{gray!25}Gb & Ab & B & C \\ \hline
    Fis & Gis & \cellcolor{gray!25}Ais & \cellcolor{gray!25}B & Cis & Dis & F \\ \hline
    \end{tabularx}
\end{table}