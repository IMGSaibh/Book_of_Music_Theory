\subsection{Akkord Typen}

Dein Songwriting wird massiv davon profitieren, wenn du ein wenig deiner Energie darauf verwendest, ein paar grundlegende Konzepte der Musiktheorie zu erlernen. Eine absolut essentielle Fähigkeit besteht darin, Akkorde bilden zu können. Das Bilden von Akkorden ist eine Grundlage der Musiktheorie, mit der du komplett verändern kannst, wie du Musik hörst, verstehst und schreibst. \textbf{Akkorde bestehen aus mindestens zwei harmonischen Tönen, die gleichzeitig gespielt werden. Die meisten grundlegenden Akkorde bestehen aus drei Tönen.} Akkorde werden \textbf{von ihrem Grundton aus gebildet.} Die restlichen Töne eines Akkords werden durch die Qualität des Akkords bestimmt.

Das Bilden von Akkorden und Akkordfolgen ist simpel, wenn du erst einmal die nötigen Grundlagen verstanden hast. Doch um Akkorde bilden zu können, musst du zunächst wissen, was Intervalle sind. Intervalle sind die Abstände zwischen Tönen. \textbf{Akkorde sind schlicht und ergreifend unterschiedliche Kombinationen von Intervallen. Wie du gleich sehen wirst, kann die kleinste Veränderung der Intervalle eines Akkords massive Veränderungen im Akkord hervorrufen.} Um Intervalle zu verstehen, musst du in Halbtonschritten denken. Die Halbtonschritte haben wir bereits in den ersten Kapiteln behandelt. Lies dort gern noch mal nach falls du dir unsicher bist. Um alles einfach zu halten, zeige ich dir zuerst wie du Dur-Akkorde ausgehend von der Tonart C bildest. In der westlichen Musik gibt es viele verschiedene Akkordtypen, die aus verschiedenen Kombinationen von Intervallen bestehen. Hier sind einige der häufigsten und wichtigsten Akkord Typen.


\subsubsection{Dreiklänge}
\subsubsection*{Dur-Dreiklang}
Dur-Dreiklänge werden gebildet, indem man \textbf{dem Grundton eine große Terz und eine Quinte hinzufügt.} Der Grundton ist übrigens der Ton, mit dem der Akkord beginnt (in diesem Beispiel ist der Grundton C). Die große Terz ist der Abstand zwischen dem Grundton und den vier nächsten Halbtönen darüber. Da C unser Grundton ist, ist E der Ton, der eine große Terz darüber liegt. Der dritte Ton ist eine Quinte, also sieben Halbtonschritte, über dem Grundton. In unserem Beispiel wäre das dementsprechend der Abstand zwischen C und G. Wenn du diese drei Töne kombinierst, erhältst du einen C-Dur-Akkord.

\begin{table}[H]
    \caption{Dur Dreiklang}
    \begin{tabularx}{\textwidth}{|*{8}{>{\hsize=.1\hsize}X|}}
    \hline
    \cellcolor{gray!25}1. Ton & 2. Ton & \cellcolor{gray!25}3. Ton & 4. Ton & \cellcolor{gray!25}5. Ton & 6. Ton & 7. Ton & 8. Ton \\ \hline
    \cellcolor{gray!25}C & D & \cellcolor{gray!25}E & F & \cellcolor{gray!25}G & A & B & C \\ \hline  
    \end{tabularx}
\end{table}

Ich gebe dir hier ein paar Beispiele wie man den C-Dur Akkorde auf der Gitarre spielen kann. Ich möchte nur darauf hinweisen, dass es besiepielswiese oft vorkommt den oktavierten Grundton eines Akkords mitzuspielen. Schau dir dazu das erste Akkorddiagramm C-Dur an. Dort haben wir auf der A-Saite (2 Seite von links) im dritten Bund den Ton C und auf der H-Saite (2 Seite von rechts) den Ton C im ersten Bund.

% \gtab{chord name}{fret:strings:fingering}
\gtab{C-Dur}{032010:032010}
\gtab{\raisebox{5pt}{\quad C-Dur}}{3fr:(113331):002340}
\gtab{\raisebox{0pt}{\quad C-Dur}}{3fr:XX3331:0023410}
\gtab{\raisebox{0pt}{\quad C-Dur}}{5fr:XX(1114):001113}
\gtab{\raisebox{0pt}{\quad C-Dur}}{8fr:X3321X:034210}
\gtab{\raisebox{0pt}{\quad C-Dur}}{10fr:3(11)300:211300}

\subsubsection*{Moll-Dreiklang}
Moll-Dreiklänge klingen völlig anders als die fröhliche, aufgelöste Natur von Dur-Akkorden, und das obwohl sie sich nur in einem Ton von Dur unterscheiden. Moll Akkorde werden genutzt, um alle möglichen Arten von Emotionen musikalisch auszudrücken. \textbf{Moll-Akkorde werden gebildet, indem man dem Grundton eine kleine Terz (drei Halbtöne) und eine Quinte hinzufügt.}

\begin{table}[H]
    \caption{Moll Dreiklang}
    \begin{tabularx}{\textwidth}{|*{8}{>{\hsize=.1\hsize}X|}}
    \hline
    \cellcolor{gray!25}1. Ton & 2. Ton & \cellcolor{gray!25}3. Ton & 4. Ton & \cellcolor{gray!25}5. Ton & 6. Ton & 7. Ton & 8. Ton \\ \hline
    \cellcolor{gray!25}C & D & \cellcolor{gray!25}Dis/Es & F & \cellcolor{gray!25}G & Gis/As & Bb & C \\ \hline  
    \end{tabularx}
\end{table}

Ich gebe dir hier wieder ein paar Beispiele wie man den C-Moll Akkorde auf der Gitarre spielen kann.

% \gtab{chord name}{fret:strings:fingering}
\gtab{C-Moll}{X3101X:041020}
\gtab{\raisebox{0pt}{\quad C-Moll}}{3fr:X(13321):004320}
\gtab{\raisebox{0pt}{\quad C-Moll}}{3fr:XXX321:000321}
\gtab{\raisebox{5pt}{\quad C-Moll}}{8fr:(133111):034000}
\gtab{\raisebox{0pt}{\quad C-Moll}}{8fr:X33(11)X:034110}
\gtab{\raisebox{0pt}{\quad C-Moll}}{8fr:X3X2X1:030201}

\subsubsection*{Verminderter Dreiklang}
Verminderte Dreiklänge sorgen für einen spannungsgeladenen, dissonanten Klang. Diese Akkorde werden gebildet, indem man dem Grundton eine kleine Terz und einen Tritonus hinzufügt. Ein Tritonus besteht aus sechs Halbtönen. \textbf{Man kann Gedanklich auch einen Moll-Akkord hernehmen und dessen Quinte um einen Halbtonschritt vermindern.} Im Jazz jedoch wird hier noch die kleine Septime ergänzt. In der Regel werden verminderte Akkorde als Vierklang gespielt. Also meistens mit der verminderten Septime (die kleine Septime um einen Halbton vermindert).

\begin{table}[H]
    \caption{Verminderter Dreiklang}
    \scriptsize
    \begin{tabularx}{\textwidth}{|X|X|p{1.4cm}|X|p{1.4cm}|X|p{1.4cm}|p{1.4cm}|X|X|}
    \hline
    1. Ton & 2. Ton & 3. Ton & 4. Ton & $\frac{1}{2}$Ton & 5. Ton & 6. Ton & $\frac{1}{2}$Ton & 7. Ton & 8. Ton \\ \hline
    \cellcolor{gray!25}C & D & \cellcolor{gray!25}Dis/Es & F & \cellcolor{gray!25}Fis/Ges & G & Gis/As & A/Bbb & Bb & C \\ \hline  
    \cellcolor{gray!25}C & D & \cellcolor{gray!25}Dis/Es & F & \cellcolor{gray!25}Fis/Ges & G & Gis/As & \cellcolor{gray!25}A/Bbb & Bb & C \\ \hline  
    \end{tabularx}
\end{table}

Ich gebe dir hier wieder ein paar Beispiele wie man verminderte C Akkorde auf der Gitarre spielen kann. Neben den verminderten Dreiklang Akkorden, gibt es noch die verminderten Vierklang Akkorde. \textbf{Bei diesen wird zum Beispiel zum verminderten Cdim eine verminderte kleine Septime hinzugenommen.}


\gtab{\raisebox{5pt}{\quad Cdim7}}{X3X(242):020131}
\gtab{\raisebox{5pt}{\quad Cdim7}}{2fr:X2313X:023140}
\gtab{\raisebox{5pt}{\quad Cdim7}}{5fr:X2313X:231400}
\gtab{\raisebox{5pt}{\quad Cdim7}}{8fr:X2313X:023140}
\gtab{\raisebox{10pt}{\quad Cdim7}}{8fr:(123131):023040}
\gtab{\raisebox{10pt}{\quad Cdim7}}{8fr:(12313)X:023040}
\gtab{\raisebox{10pt}{\quad Cdim7}}{10fr:XX(121)2:001213}


\subsubsection*{Übermäßiger Dreiklang}
Übermäßige Dreiklänge klingen bizarr und beunruhigend, wie ein Soundtrack eines Science-Fiction-Films. Von allen Grundakkorden ist der übermäßige Akkord derjenige, der am seltensten in der Musik vorkommt. Übermäßige Akkorde werden genauso gebildet wie simple Dur-Akkorde, jedoch mit einer übermäßigen Quinte. \textbf{Der C-Dur-Akkord beinhaltet die Töne C, E und G, der übermäßige C-Akkord beinhaltet daher C, E und G\#.}

\begin{table}[H]
    \caption{Übermäßige Akkorde}
    \scriptsize
    \begin{tabularx}{\textwidth}{|X|X|X|X|p{1.4cm}|X|X|X|X|}
    \hline
    1. Ton & 2. Ton & 3. Ton & 4. Ton & 5. Ton & $\frac{1}{2}$Ton & 6. Ton & 7. Ton & 8. Ton \\ \hline
    \cellcolor{gray!25}C & D & \cellcolor{gray!25}E & F & G & \cellcolor{gray!25}Gis/As & A & B & C \\ \hline  
    \end{tabularx}
\end{table}

\gtab{\raisebox{5pt}{\quad C+}}{X3211X:032110}
\gtab{\raisebox{5pt}{\quad C+}}{fr5:4321XX:432100}
\gtab{\raisebox{5pt}{\quad C+}}{fr6:321XXX:321000}
\gtab{\raisebox{5pt}{\quad C+}}{fr6:3214XX:321400}
\gtab{\raisebox{5pt}{\quad C+}}{fr8:(1432X1):143201}
\gtab{\raisebox{5pt}{\quad C+}}{fr8:XX3221:004231}
\gtab{\raisebox{5pt}{\quad C+}}{fr9:XX2(11)X:002110}
\gtab{\raisebox{5pt}{\quad C+}}{fr9:XX2(11)X:002110}

\subsubsection{Vierklänge oder auch Septakkorde}

\subsubsection*{Dur-Septakkord}
Diese Akkordtypen bestehen aus einem Dur-Dreiklang plus einer \textbf{große Septime}.

\subsubsection*{Dur-Moll-Septakkord oder auch Dominantseptakkord}
Der Dur-Moll-Septakkord oder auch Dominantseptakkord bekommt seinen Namen weil er aus dem Grundton, der großen Terz, der reinen Quinte und der kleinen Septime zusammengesetzt ist.
Dur-Dreiklang plus \textbf{kleine Septime}

\subsubsection*{Moll-Septakkord}
Der Moll-Septakkord unterscheidet sich vom Dur-Moll-Septakkord nur durch die Terz. Wir verwenden bei diesem Akkord daher den Grundton, die kleine Terz, die reine Quinte und die kleine Septime.
\textbf{Moll-Dreiklang plus kleine Septime}

\subsubsection*{Halbverminderter Septakkord}
verminderter Dreiklang plus große Septime.
\subsubsection*{Verminderter Septakkord}
verminderter Dreiklang plus verminderte Septime



\subsubsection{Erweiterte Akkorde}
Diese Akkord Typen enthalten Töne, die über die Septime hinaus gehen, wie None, Undezime und Tredezime.
\subsubsection*{None-Akkorde}
\subsubsection*{Undezime-Akkorde}
\subsubsection*{Tredezime-Akkorde}

\subsubsection{Sonstige Akkorde}
\subsubsection*{Suspendierte Akkorde}
\subsubsection*{Quartakkorde}
\subsubsection*{Cluster-Akkorde}
\subsubsection*{Slash-Akkorde} 
in Akkord, welcher jedoch einen anderen Basston als den Grundton hat. Die Bassnote kann die Terz oder Quinte sein, vielfach wird jedoch auch ein anderer Ton genommen, welcher meist aus der diatonischen Tonleiter stammt. 

Dies ist keine vollständige Liste aller Akkord Typen, da es viele Variationen und Kombinationen von Akkorden gibt, insbesondere in modernen Musik Stilen wie Jazz und zeitgenössischer, klassischer Musik. Akkord Strukturen können auch durch Hinzufügen, Weglassen oder Verändern von Tönen verändert werden, um einzigartige Klänge und Texturen zu erzeugen.