\subsection{Harmonie}
Harmonien sind wichtige Elemente eines Songs. Sie ist allerdings nicht immer leicht umzusetzen. 
Doch es gibt ein paar Elemente eines Songs, die das Songwriting voranbringen – die Harmonik gehört dazu. 
Egal ob fortgeschrittene*r Musiker*in oder vollkommener Neuling – musiktheoretisches Wissen bietet massive Vorteile.

Harmonien entstehen immer dann, wenn zwei oder mehrere Töne gleichzeitig gespielt werden. \textbf{Harmonik} kann 
sich auf das \textbf{Arrangement individueller Töne bzw. Tonstufen zu einem Akkord sowie auf die allgemeine Akkordstruktur 
eines Musikstücks beziehen}. Doch in der Musiktheorie bezieht sich Harmonik normalerweise auf das Bilden 
von Akkorden, die Eigenschaften von Akkorden und Akkordfolgen.

Harmonik wird durch römische Ziffern repräsentiert. Der Name eines Akkords wird durch eine römische Ziffer
ersetzt. Und zwar \textbf{die Ziffer die der Tonstufe seines Grundtons entspricht}. Römische Ziffern geben an, 
zu welcher harmonischen Kategorie ein Akkord gehört. Und Akkord-Kategorien bestimmen die Funktion eines 
Akkords innerhalb eines Musikstücks. In der tonalen Musik gibt es drei funktionale Kategorien:

\begin{itemize}
    \item \textbf{Tonika}: Tonika-Akkorde sind Ruhezonen, bei denen einem die harmonische Aktion eines 
    Songs am stabilsten vorkommt.
    \item \textbf{Dominante}: Dominant-Akkorde stellen eine Art Gegenteil zu den Tonika-Akkorden dar. 
    Bei diesem Akkord will die vierte und siebte Tonleiterstufe auf natürliche Weise nach unten 
    (vierte-dritte) und nach oben (siebte-Tonika) rücken, um den Tonika-Akkord aufzulösen.
    \item \textbf{Prädominante}: Die anderen Akkorde in dem harmonischen Vokabular sind die Prädominanten, 
    mit denen die Lücke zwischen Stufe I und Stufe V überbrückt wird.
\end{itemize}

Hier vielleicht der Quintenzirkel und ein Beispiel welche Akkorde gut zusammen passen